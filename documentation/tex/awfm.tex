%%%%%%%%%%%%%%%%%%%%%%%%%%%%%%%%%%%%%%%%%%%%%%%%%%%
%% LaTeX book template                           %%
%% Author:  Amber Jain (http://amberj.devio.us/) %%
%% License: ISC license                          %%
%%%%%%%%%%%%%%%%%%%%%%%%%%%%%%%%%%%%%%%%%%%%%%%%%%%

\documentclass[a4paper,11pt]{book}
\usepackage[T1]{fontenc}
\usepackage[utf8]{inputenc}
\usepackage{lmodern}
%%%%%%%%%%%%%%%%%%%%%%%%%%%%%%%%%%%%%%%%%%%%%%%%%%%%%%%%%
% Source: http://en.wikibooks.org/wiki/LaTeX/Hyperlinks %
%%%%%%%%%%%%%%%%%%%%%%%%%%%%%%%%%%%%%%%%%%%%%%%%%%%%%%%%%
\usepackage{hyperref}
\usepackage{graphicx}
\usepackage[english]{babel}

%%%%%%%%%%%%%%%%%%%%%%%%%%%%%%%%%%%%%%%%%%%%%%%%%%%%%%%%%%%%%%%%%%%%%%%%%%%%%%%%
% 'dedication' environment: To add a dedication paragraph at the start of book %
% Source: http://www.tug.org/pipermail/texhax/2010-June/015184.html            %
%%%%%%%%%%%%%%%%%%%%%%%%%%%%%%%%%%%%%%%%%%%%%%%%%%%%%%%%%%%%%%%%%%%%%%%%%%%%%%%%
%\newenvironment{dedication}
%{
%   \cleardoublepage
%   \thispagestyle{empty}
%   \vspace*{\stretch{1}}
%   \hfill\begin{minipage}[t]{0.66\textwidth}
%   \raggedright
%}
%{
%   \end{minipage}
%   \vspace*{\stretch{3}}
%   \clearpage
%}

%%%%%%%%%%%%%%%%%%%%%%%%%%%%%%%%%%%%%%%%%%%%%%%%
% Chapter quote at the start of chapter        %
% Source: http://tex.stackexchange.com/a/53380 %
%%%%%%%%%%%%%%%%%%%%%%%%%%%%%%%%%%%%%%%%%%%%%%%%
\makeatletter
\renewcommand{\@chapapp}{}% Not necessary...
\newenvironment{chapquote}[2][2em]
  {\setlength{\@tempdima}{#1}%
   \def\chapquote@author{#2}%
   \parshape 1 \@tempdima \dimexpr\textwidth-2\@tempdima\relax%
   \itshape}
  {\par\normalfont\hfill--\ \chapquote@author\hspace*{\@tempdima}\par\bigskip}
\makeatother

%%%%%%%%%%%%%%%%%%%%%%%%%%%%%%%%%%%%%%%%%%%%%%%%%%%
% First page of book which contains 'stuff' like: %
%  - Book title, subtitle                         %
%  - Book author name                             %
%%%%%%%%%%%%%%%%%%%%%%%%%%%%%%%%%%%%%%%%%%%%%%%%%%%

% Book's title and subtitle
\title{\Huge \textbf{Analytic Wellfield Model}}
% Author
\author{\textsc{Alan Lewis}\thanks{\url{www.github.com/geouke/awfm}}}


\begin{document}

\frontmatter
\maketitle

%%%%%%%%%%%%%%%%%%%%%%%%%%%%%%%%%%%%%%%%%%%%%%%%%%%%%%%%%%%%%%%
% Add a dedication paragraph to dedicate your book to someone %
%%%%%%%%%%%%%%%%%%%%%%%%%%%%%%%%%%%%%%%%%%%%%%%%%%%%%%%%%%%%%%%
%\begin{dedication}
%Dedicated to Calvin and Hobbes.
%\end{dedication}

%%%%%%%%%%%%%%%%%%%%%%%%%%%%%%%%%%%%%%%%%%%%%%%%%%%%%%%%%%%%%%%%%%%%%%%%
% Auto-generated table of contents, list of figures and list of tables %
%%%%%%%%%%%%%%%%%%%%%%%%%%%%%%%%%%%%%%%%%%%%%%%%%%%%%%%%%%%%%%%%%%%%%%%%
\tableofcontents
\listoffigures
\listoftables

\mainmatter

%%%%%%%%%%%
% Preface %
%%%%%%%%%%%
\chapter*{Introduction}

\begin{chapquote}{Author's name, \textit{Source of this quote}}
``This is a quote and I don't know who said this.''
\end{chapquote}

Lorem ipsum dolor sit amet, consectetur adipiscing elit. Duis risus ante, auctor et pulvinar non, posuere ac lacus. Praesent egestas nisi id metus rhoncus ac lobortis sem hendrerit. Etiam et sapien eget lectus interdum posuere sit amet ac urna.

%%%%%%%%%%%%%%%%%%%%%%%%%%%%%%%%%%%%
% Give credit where credit is due. %
% Say thanks!                      %
%%%%%%%%%%%%%%%%%%%%%%%%%%%%%%%%%%%%
%\section*{Acknowledgements}
%\begin{itemize}
%\item A special word of thanks goes to Professor Don Knuth\footnote{\url{http://www-cs-faculty.stanford.edu/~uno/}} (for \TeX{}) and Leslie Lamport\footnote{\url{http://www.lamport.org/}} (for \LaTeX{}).
%\item I'll also like to thank Gummi\footnote{\url{http://gummi.midnightcoding.org/}} developers and LaTeXila\footnote{\url{http://projects.gnome.org/latexila/}} development team for their awesome \LaTeX{} editors.
%\item I'm deeply indebted my parents, colleagues and friends for their support and encouragement.
%\end{itemize}
%\mbox{}\\
%\mbox{}\\
%\noindent Amber Jain \\
%\noindent \url{http://amberj.devio.us/}

%%%%%%%%%%%%%%%%
% NEW CHAPTER! %
%%%%%%%%%%%%%%%%
\chapter{Mathematical Model}

\section{Forward Model}
The forward model estimates water level elevation ($h$) at any point in time from three components: 
recoverable water level ($h_0$), aquifer drawdown ($s_{aq}$), and well loss ($s_{wl}$).

$$h = h_0 - s_{aq} - s_{wl}$$

\emph{Recoverable water level} is defined as the elevation to which water level in a well
would recover if all pumping was stopped. It may be modeled as either steady state or 
transient. In the steady-state case, a recoverable water level does not change through time
and is equivalent to the static water level appearing in many analytical groundwater solutions.
In the transient case, recoverable water level is modeled as a linear function through time.


$$h_0(t) = h_0(0) + \frac{\Delta h}{\Delta t}t$$


The \emph{aquifer drawdown model} is an analytical solution for aquifer drawdown that 
can accomodate multiple pumping rates across multiple wells. Currently, the AWFM supports
the Theis and Hantush-Jacob solutions.

\emph{Well loss} is modeled using Jacob's (ref?) empirical formula:

$$s_{wl} = CQ^2 + BQ$$

$Q$ is the instantaneous pumping rate at a given well. Coefficients $B$ and $C$
represent well losses due to laminar and turbulant flow, respectively. 
$B$ and $C$ can either be constant through time, or modeled using the linear
functions:

$$B(t) = B(0) + \frac{\Delta B}{\Delta t}t$$
$$C(t) = C(0) + \frac{\Delta C}{\Delta t}t$$


\section{Parameter Estimation}

\subsection{Aquifer Parameters}

\subsection{Recoverable Water Level(s) and Well-Loss Coefficient(s)}

\chapter{Using the Graphical Interface}

\section{Creating a New Model}

\section{Units}

The units dialog is used for specifying three unit types in the model:
length, time, and discharge. The discharge unit applies to pumping rates
\footnote{In reality, discharge may be expressed in units of length and time. 
Here, discharge units are specified separately to allow plotting in units familiar to the user.},
while length and time units apply to everything else.

\section{Data Importing}

\subsection{CSV}

\subsection{Excel}

\subsection{SQLite}

\section{Timeseries Processing}

The timeseries dialog, which is used for importing and viewing
pumping rates and observed water levels, has a handful of useful 
functions detailed below.

\subsection{Scale}

\emph{Scale} here is used in the mathematical sense of multiplying
a value by a scalar. This function is most useful for converting 
between units, but could also be used to model what-if scenarious. 
For example, it is possible to see the effect of increasing or 
decreasing discharge volumes by a percent.

\subsection{Translate}

\emph{Tranlation} is also used in the mathematical sense of adding
some scalar to a value. This function is useful if observed water
levels at multiple wells were all taken relative to their own piezometer
elevations and need to be converted to a consistent elevation.

Another application of the translation function is to adjust $t_0$.
Data imported from Excel may have a $t_0$ in the early 1900s, whereas
it may be more useful to have a $t_0$ when production began at the wellfield.

\subsection{Erroneous/Missing Data}

The data import utility will set missing row values to -9999. These need
to be dealt with before the forward model can produce meaningful results.

One option is to simply remove missing data. In the context of pumping
rates, this means that the value at the missing data point is effectively
equal to the value at the previously measured data point:

<insert table>

Another option is to perform a linear interpolation through the missing
data point, which may be desirable for approximating missing water levels.

\subsection{Project Onto Line}

This operation performs a piece-wise linear interpolation, which is
useful for data reduction or simply for obtaining a data set with a 
constant time step for input into a numerical model.

\subsection{Range Constraints}

\subsection{Data Reduction}

\chapter{Examples}

\section{Simple Pumping Test}

\section{Analysis With Realistic Data}


\end{document}
